\documentclass{article}
\usepackage[utf8]{inputenc}
\usepackage[russian]{babel}
\usepackage{amsmath}
\usepackage{graphicx}
\usepackage{float}
\usepackage{hyperref}
\usepackage{geometry}
\usepackage{wrapfig}

\geometry{
 a4paper,
 total={170mm,257mm},
 left=20mm,
 top=20mm,
}

\title{\vspace{-2em}Определение концентрации меди на основе\\ спектральных данных XRF}
\date{\today}

\begin{document}

\maketitle

\begin{abstract}
	В данной работе исследуется возможность определения концентрации меди (Cu) в образцах на основе данных рентгенофлуоресцентного анализа (XRF). Разработан конвейер обработки данных, включающий извлечение физически интерпретируемых признаков (интенсивностей спектральных линий) из сырых спектров, статистическую предобработку и обучение линейной модели с $L_1$-регуляризацией (LASSO). Для настройки модели использовался метод проксимального градиентного спуска и стратифицированная кросс-валидация. 
\end{abstract}

\section{Введение}
Целью работы является построение регрессионной модели $f(S) \to C_{Cu}$, где $S$ — исходный энергетический спектр, а $C_{Cu}$ — концентрация меди. Данные представляют собой набор спектров (1024 канала) и соответствующих лабораторных значений концентраций. Сложность задачи обусловлена небольшим исходным набором данных (243 спектра), необходимостью корректного отбора признаков и наличием фона.

\section{Методология}

\begin{figure}
	\centering
	\includegraphics[width=1.0\textwidth]{assets/spectrum_example.png}
	\caption{Пример исходных данных}
\end{figure}

\subsection{Извлечение признаков}
Вместо подачи сырых отсчетов (counts) всех 1024 каналов в модель, был применен подход, основанный на физике процесса. Для каждого химического элемента, определенного в файле \texttt{Elements.txt} (S, Ag, Ar, Ca, Ti, Cr, Fe, Ni, Cu, Zn, Pb, Kr), вычислялась интегральная интенсивность пика.

Алгоритм обработки (\texttt{process\_spectra\_intensities}):
\begin{enumerate}
	\item \textbf{Определение ROI (Region of Interest):} Для каждой спектральной линии (например, $K\alpha, K\beta, L\alpha$) определены границы каналов $[L, R]$.
	\item \textbf{Вычитание фона:} Реализован метод \texttt{net\_peak\_area\_neighbor\_aware}. Для оценки фона используются интервалы слева и справа от пика. Фон аппроксимируется линейно между средними значениями в левой и правой фоновых областях.
	\item \textbf{Нормировка на live time:} Полученные чистые площади пиков (net counts) нормируются на время набора спектра ($LIVE\_TIME$), преобразуясь в интенсивности (cps — counts per second).
\end{enumerate}
Итоговый набор данных $X$ состоит из интенсивностей линий различных элементов.

\subsection{Предобработка данных}
Для улучшения стабильности и качества обучения были применены следующие этапы:
\begin{itemize}
	\item \textbf{Стратифицированное разбиение:} Целевая переменная (концентрация) была дискретизирована на 3 категории (\textit{small, mid, large}) по квантилям 0.35 и 0.70. Разбиение на обучающую (80\%) и тестовую (20\%) выборки проводилось стратифицированно по этим категориям, чтобы гарантировать репрезентативность теста.
	\item \textbf{Логарифмирование целевой переменной:} Для работы с распределением концентраций и уменьшения гетероскедастичности применено преобразование $y' = \ln(y + \epsilon)$. Логарифмическое преобразование также позволяет избежать отрицательных предсказаний, что было бы не верно для концентраций.
	\item \textbf{Клиппинг выбросов} Для признаков (интенсивностей) в обучающей выборке применено ограничение значений (clipping) по 5-му и 95-му перцентилям. Это позволяет снизить влияние аномальных выбросов в спектрах на коэффициенты линейной модели.
	\item \textbf{Стандартизация:} Все признаки приведены к нулевому среднему и единичной дисперсии (\texttt{StandardScaler}).
\end{itemize}

\begin{figure}[h!]
	\centering
	\includegraphics[width=1.0\textwidth]{assets/clipping.png}
\end{figure}

\section{Моделирование}

\subsection{Модель LASSO}
В качестве предиктивной модели выбрана линейная регрессия с $L_1$-регуляризацией (LASSO). Данный выбор обусловлен необходимостью отбора признаков: $L_1$-штраф зануляет коэффициенты при незначимых элементах, оставляя только те, которые физически коррелируют с содержанием меди (например, сама линия Cu и, возможно, элементы матрицы, влияющие на поглощение).

Целевая функция минимизации:
$$ \min_{w} \left( \frac{1}{2N} ||Xw - y||_2^2 + \alpha ||w||_1 \right) $$

\subsection{Оптимизация}
Для обучения модели был реализован кастомный солвер на основе \textbf{проксимального градиентного спуска (Proximal Gradient Descent)}:
\begin{enumerate}
	\item Шаг градиентного спуска по гладкой части функции потерь (MSE).
	\item Применение оператора мягкого порога (Soft Thresholding) для учета негладкого $L_1$-члена:
	      $$ S_\tau(z) = \text{sign}(z) \cdot \max(|z| - \tau, 0) $$
\end{enumerate}

Оптимальный параметр регуляризации $\alpha$ подбирался с помощью стратифицированной кросс-валидации (5 фолдов) на обучающей выборке. Критерием выбора являлась минимизация среднеквадратичной ошибки (MSE) на валидационных фолдах.

\section{Результаты}

\subsection{Вклад признаков:}

\begin{wrapfigure}[15]{r}{0.45\textwidth}
  \centering
  \includegraphics[width=\linewidth]{assets/selected_features.png}
\end{wrapfigure}

Несмотря на ограниченный объем обучающей выборки, распределение весов регрессии демонстрирует, что модель успешно выявила фундаментальные зависимости рентгенофлуоресцентного анализа. \\

\noindent Ключевые признаки, отобранные LASSO-регуляризацией:

\begin{itemize}
    \item \textbf{Cu\_Ka ($+0.149$):} Основной предиктор. Положительный вес подтверждает прямую зависимость между интенсивностью характеристической линии и концентрацией элемента.
    
    \item \textbf{Cu\_Kb ($0.000$):} Вес занулен. Это свидетельствует о корректной работе регуляризатора: линия $K\beta$ жестко коррелирована с $K\alpha$ и не несет дополнительной информации. Исключение мультиколлинеарности повышает устойчивость модели.
\end{itemize}
    
\begin{itemize}
    \item \textbf{Fe\_Ka ($-0.145$):} Значительный отрицательный вес. Данный эффект объясняется физикой взаимодействия излучения с веществом. Энергия фотонов Cu $K\alpha$ ($8.04$ кэВ) превышает край поглощения железа Fe $K$-edge ($7.11$ кэВ), что приводит к сильному поглощению излучения меди атомами железа. Отрицательный коэффициент в линейной модели компенсирует это явление, занижая видимый сигнал меди при высоком содержании железа.    
    \item \textbf{Пики рассеяния NKr ($-0.069$) и Kr ($-0.033$):} Отрицательные веса коррелируют с теорией. Интенсивность рассеянного излучения обратно пропорциональна среднему атомному номеру $Z$ образца. Высокая концентрация металлов (Cu, Fe) повышает $Z$ и подавляет рассеяние.
\end{itemize}

\subsection{Итоговые результаты}

На отложенной тестовой выборке среднеквадратичная ошибка cоставила RMSE = 1.13

\begin{figure}[h!]
	\centering
	\includegraphics[width=1.0\textwidth]{assets/target_prediction.png}
\end{figure}

Не смотря на использование стратифицированного разбиения, наблюдаем обратную пропорциональность точности предсказаний от величины концентрации. Такое поведение может быть обусловлено использованием слишком простой модели, недостаточной предобработкой исходных данных, а также небольшой величиной выборки.

\end{document}
